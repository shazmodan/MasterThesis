\section{Merge Sort}

%\begin{figure}[h]
	\lstset{style=sharpc}
	\begin{lstlisting}
using System;
using System.Text;

public class CSharpMergeSort
{
	public static int[] Sort(int [] array)
	{
		if (array.Length > 1)
		{
			int elementsInA1 = array.Length / 2;
			int elementsInA2 = array.Length - elementsInA1;
			int[] arr1 = new int[elementsInA1];
			int[] arr2 = new int[elementsInA2];

			for (int y = 0; y < elementsInA1; y++)
			{
				arr1[y] = array[y];
			}


			for (int x = elementsInA1; x < elementsInA1 + elementsInA2; x++)
			{
				arr2[x - elementsInA1] = array[x];
			}


			arr1 = Sort(arr1);
			arr2 = Sort(arr2);

			int i = 0, j = 0, k = 0;

			while (arr1.Length != j && arr2.Length != k)
			{
				if (arr1[j] <= arr2[k])
				{
					array[i] = arr1[j];
					i++;
					j++;
				}
				else
				{
					array[i] = arr2[k];
					i++;
					k++;
				}
			}

			while (arr1.Length != j)
			{
				array[i] = arr1[j];
				i++;
				j++;
			}
			while (arr2.Length != k)
			{
				array[i] = arr2[k];
				i++;
				k++;
			}
		}
		return array;
	}

	public static String HandleInputOutput(String input)
	{
		// Organize input.
		String[] splitted = input.Split(' ');
		int[] array = new int[splitted.Length];
		for (int i = 0; i < splitted.Length; i++)
		{
			array[i] = int.Parse(splitted[i]);
		}


		// Sort the array (using mergesort)
		array = Sort(array);

		
		// Prepare output.
		StringBuilder builder = new StringBuilder();
		for (int i = 0; i < array.Length; i++)
		{
			builder.Append(array[i] + " ");
		}
		return builder.ToString().TrimEnd();
	}
	
	public static void Main(String [] args){}
}

	\end{lstlisting}
	%\caption{To get the other tests simply replace the operator.}
	%\label{fig:language_overhead}
%\end{figure}

%\begin{figure}[h]
	\lstset{style=java}
	\begin{lstlisting}

public class JavaMergeSort{
	
	public int[] Sort(int array[])  {
		if(array.length > 1)  	{
			int elementsInA1 = array.length/2;
			int elementsInA2 = array.length - elementsInA1;
			int arr1[] = new int[elementsInA1];
			int arr2[] = new int[elementsInA2];

			for(int i = 0; i < elementsInA1; i++)
				arr1[i] = array[i];

			for(int i = elementsInA1; i < elementsInA1 + elementsInA2; i++)
				arr2[i - elementsInA1] = array[i];

			arr1 = Sort(arr1);
			arr2 = Sort(arr2);

			int i = 0, j = 0, k = 0;

			while(arr1.length != j && arr2.length != k) {
				if(arr1[j] <= arr2[k]) {
					array[i] = arr1[j];
					i++;
					j++;
				} else {
					array[i] = arr2[k];
					i++;
					k++;
				}
			}

			while(arr1.length != j) {
				array[i] = arr1[j];
				i++;
				j++;
			}
			while(arr2.length != k) {
				array[i] = arr2[k];
				i++;
				k++;
			}
		}
		return array;
	}

	public String HandleInputOutput(String input){

		// Organize input. 
		String [] splitted = input.split(" ");
		int [] A = new int[splitted.length];
		for(int i = 0; i < splitted.length; i++){
			A[i] = Integer.parseInt(splitted[i]);
		}

		// Start merge sort.
		A = Sort(A);

		// Prepare output.
		StringBuffer buffy = new StringBuffer();
		for(int i = 0; i < A.length; i++){
			buffy.append(A[i] + " ");
		}
		return buffy.toString().trim();
	}
	
	public static void main(String[] args){}
}

	\end{lstlisting}
	%\caption{C\# code for language overhead testing.}
	%\label{fig:language_overhead}
%\end{figure}

%\begin{figure}[h]
	\lstset{style=python}
	\begin{lstlisting}
import sys
import time
import array

class PythonMergeSort:
	def Sort(self, arr):
		if len(arr) == 1:
			return arr
		
		m = len(arr) // 2
		l = self.Sort(arr[:m])
		r = self.Sort(arr[m:])

		if not len(l) or not len(r):
			return l or r
			
		result = []
		i = j = 0
		while (len(result) < len(r)+len(l)):        
			if l[i] < r[j]:
				result.append(l[i])
				i += 1
			else:
				result.append(r[j])
				j += 1            
			if i == len(l) or j == len(r):            
				result.extend(l[i:] or r[j:])
				break
			
		return result
	
	def HandleInputOutput(self, input):
		# Organise input.
		input = input.split(" ")
		array = list(map(int, input))
		
		# Start merge sort.
		derp = MergeSort()
		array = derp.Sort(array)

		# Prepare output.
		array = map(str, array)
		returnString = " ".join(array)

		return returnString

	\end{lstlisting}
	%\caption{C\# code for language overhead testing.}
	%\label{fig:language_overhead}
%\end{figure}