\section{Common Operators}

\subsection{Addition} \label{appendix:code_addition}

\subsubsection{C\#}
\lstset{style=sharpc}
\begin{lstlisting}
using System;

public class AdditionTest
{
    //int.parse() is faster than convert.ToInt32() since convert.ToInt32 performs an additional null check.
    public string AddNumbers(string input)
    {
        int sum = 0;

        string[] splitted = input.Split(' ');
        for (var i = 0; i < splitted.Length; i++)
        {
            sum += Int32.Parse(splitted[i]);
        }
        return sum.ToString();
    }

    public static void Main(){}
}

\end{lstlisting}

\subsubsection{Java}
\lstset{style=java}
\begin{lstlisting}

public class AdditionTest {

	public String AddNumbers(String input){
		int sum = 0;
		String [] splitted = input.split(" ");

		for(int i = 0; i < splitted.length; i++){
			sum += Integer.parseInt(splitted[i]);
		}

		return "" + sum;
	}

	public static void main(String[] args){}
}

\end{lstlisting}

\subsubsection{Python}
\lstset{style=python}
\begin{lstlisting}
import sys
class AdditionTest:
    def AddNumbers(self, input):
        input = input.split(" ")
        array = list(map(int, input))
        total = sum(array)
        return str(total)
	\end{lstlisting}
	%\caption{C\# code for language overhead testing.}
	%\label{fig:language_overhead}
%\end{figure}



\subsection{Subtraction}


%\begin{figure}[h]
	\lstset{style=sharpc}
	\begin{lstlisting}
using System;
public class SubtractionTest
{
    //int.parse() is faster than convert.ToInt32() since it performs an additional null check.
    public string SubtractNumbers(string input)
    {
        int sum = 0;

        string[] splitted = input.Split(' ');
        for (var i = 0; i < splitted.Length; i++)
        {
            sum -= Int32.Parse(splitted[i]);
        }
        return sum.ToString();
    }

	public static void Main(){}
}
	\end{lstlisting}
	%\caption{To get the other tests simply replace the operator.}
	%\label{fig:language_overhead}
%\end{figure}

%\begin{figure}[h]
	\lstset{style=java}
	\begin{lstlisting}
public class SubtractionTest {

	public String SubtractNumbers(String input){
		int sum = 0;
		String [] splitted = input.split(" ");

		for(int i = 0; i < splitted.length; i++){
			sum -= Integer.parseInt(splitted[i]);
		}

		return "" + sum;
	}

	public static void main(String[] args){}
}
	\end{lstlisting}
	%\caption{C\# code for language overhead testing.}
	%\label{fig:language_overhead}
%\end{figure}


%\begin{figure}[h]
	\lstset{style=python}
	\begin{lstlisting}
import operator
class SubtractionTest:
    def SubtractNumbers(self, input):
        input = input.split(" ")
        array = list(map(int, input))
        total = reduce(operator.sub, array, 1)
        return str(total)
	\end{lstlisting}
	%\caption{C\# code for language overhead testing.}
	%\label{fig:language_overhead}
%\end{figure}


\subsection{Multiplication}


%\begin{figure}[h]
	\lstset{style=sharpc}
	\begin{lstlisting}
using System;
public class MultiplicationTest
{
    //int.parse() is faster than convert.ToInt32() since it performs an additional null check.

    public string MultiplyNumbers(string input)
    {
        int sum = 1; //multiplication starts at 1, not 0.

        string[] splitted = input.Split(' ');
        for (var i = 0; i < splitted.Length; i++)
        {
            sum *= Int32.Parse(splitted[i]);
        }
        return sum.ToString();
    }

	public static void Main(String[] args){}

}
	\end{lstlisting}
	%\caption{To get the other tests simply replace the operator.}
	%\label{fig:language_overhead}
%\end{figure}

%\begin{figure}[h]
	\lstset{style=java}
	\begin{lstlisting}
public class MultiplicationTest {

	public String MultiplyNumbers(String input){
		int sum = 1;
		String [] splitted = input.split(" ");

		for(int i = 0; i < splitted.length; i++){
			sum *= Integer.parseInt(splitted[i]);
		}

		return "" + sum;
	}

	public static void main(String[] args){}
}
	\end{lstlisting}
	%\caption{C\# code for language overhead testing.}
	%\label{fig:language_overhead}
%\end{figure}


%\begin{figure}[h]
	\lstset{style=python}
	\begin{lstlisting}
import operator
class MultiplicationTest:
    def MultiplyNumbers(self, input):
        input = input.split(" ")
        array = list(map(int, input))
        total = reduce(operator.mul, array, 1)
        return str(total)
	\end{lstlisting}
	%\caption{C\# code for language overhead testing.}
	%\label{fig:language_overhead}
%\end{figure}


\subsection{Division}


%\begin{figure}[h]
	\lstset{style=sharpc}
	\begin{lstlisting}
using System;

public class DivisionTest
{
    //int.parse() is faster than convert.ToInt32() since it performs an additional null check.

    public string DivideNumbers(string input)
    {
        float sum = 1; //divide by zero is a bad idea

        string[] splitted = input.Split(' ');
        for (var i = 0; i < splitted.Length; i++)
        {
            sum /= Int32.Parse(splitted[i]);
        }
        return sum.ToString();
    }

	public static void Main(String[] args){}

}

	\end{lstlisting}
	%\caption{To get the other tests simply replace the operator.}
	%\label{fig:language_overhead}
%\end{figure}

%\begin{figure}[h]
	\lstset{style=java}
	\begin{lstlisting}
public class DivisionTest {

	public String DivideNumbers(String input){
		float sum = 1;
		String [] splitted = input.split(" ");

		for(int i = 0; i < splitted.length; i++){
			sum /= Integer.parseInt(splitted[i]);
		}

		return "" + sum;
	}

	public static void main(String[] args){}
}
	\end{lstlisting}
	%\caption{C\# code for language overhead testing.}
	%\label{fig:language_overhead}
%\end{figure}


%\begin{figure}[h]
	\lstset{style=python}
	\begin{lstlisting}
import operator
class DivisionTest:
    def DivideNumbers(self, input):
        input = input.split(" ")
        array = list(map(int, input))
        total = reduce(operator.div, array, 1)
        return str(total)
	\end{lstlisting}
	%\caption{C\# code for language overhead testing.}
	%\label{fig:language_overhead}
%\end{figure}



\subsection{Modulo}

%\begin{figure}[h]
	\lstset{style=sharpc}
	\begin{lstlisting}
using System;

public class ModuloTest
{
    public string ModuloNumbers(string input)
    {
        double sum = double.MaxValue;

        string[] splitted = input.Split(' ');
        for (var i = 0; i < splitted.Length; i++)
        {
            sum %= Double.Parse(splitted[i]);
        }
        return sum.ToString();
    }

	public static void Main(String[] args){}
}
	\end{lstlisting}
	%\caption{To get the other tests simply replace the operator.}
	%\label{fig:language_overhead}
%\end{figure}

%\begin{figure}[h]
	\lstset{style=java}
	\begin{lstlisting}
public class ModuloTest {

	public String ModuloNumbers(String input){
		double sum = Double.MAX_VALUE;
		String [] splitted = input.split(" ");

		for(int i = 0; i < splitted.length; i++){
			sum %= Double.parseDouble(splitted[i]);
		}

		return "" + sum;
	}

	public static void main(String[] args){}
}

	\end{lstlisting}
	%\caption{C\# code for language overhead testing.}
	%\label{fig:language_overhead}
%\end{figure}


%\begin{figure}[h]
	\lstset{style=python}
	\begin{lstlisting}
import operator
class ModuloTest:
    def ModuloNumbers(self, input):
        input = input.split(" ")
        array = list(map(int, input))
        total = reduce(operator.mod, array, 1)
        return str(total)
	\end{lstlisting}
	%\caption{C\# code for language overhead testing.}
	%\label{fig:language_overhead}
%\end{figure}



