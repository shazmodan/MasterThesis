\begin{abstract}

A common challenge for software consulting companies is recruiting the right people. In the software industry, the recruitment process usually involves several steps before a contract is signed; a single job interview is rarely enough. Thus, the interview tends to involve some test to make sure that the person seeking the position is qualified. The testing procedure is usually  part of the interview or conduct at the same occasion. Interviewing applicants who are not qualified for the position might be a waste of time.

This waste can be minimized by only interviewing qualified applicants. In the software industry, qualifications are commonly asserted by letting an applicant solve programming problems. This process can be automated using an Automatic Grader. Such systems already exist on some universities today and are used extensively in various programming courses and also in programming contests.

This thesis evaluates such a system with regard to execution speed and memory consumption. It also explains how such a system can be built and attempts to demonstrate the performance differences between the supported programming languages. The system is unique in the aspect that it is built using only Microsoft .NET Framework while still supporting multiple programming languages. The supported languages are C\#, Java and Python. This support is enabled through the use of a Java byte code to CIL compiler called IKVM. Python is supported through the use of IronPython.

The results showed that C\# and Java performed almost equally in terms of execution speed, with Java being slightly behind. Python seemed to have greater performance issues than the other two. Java consumed the most memory out of the three, with Python as a close runner up.

Future work involves testing more languages and improving the system with usability in mind.


\end{abstract}