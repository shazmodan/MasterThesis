\section{Aspects to Test}
There are several ways to evaluate code. Code can be evaluated on its structure, understandability, execution speed, memory usage and others. Some programming exercises are of the nature that one would be interested in looking at execution speed and memory usage. Such is the case with many programming exercises used in KTH Kattis \cite{Kattis} and since CELINE is primarily used for securing programming knowledge of job applicants, one can expect this system to be used primarily for the recruitment of college graduates (those who have worked in the industry for a while tend to know how to program). Execution speed and memory usage are also of particular interest since they are free of any human opinion.


\subsection{Execution Speed}
The primary objective of the code should be to execute quickly. In general, the faster any code executes the more efficient the implementation is. Apart from saving time this priority enables users to compete amongst each other (using leaderboards on the website), for the shortest execution time by writing the most efficient code. It is likely that all languages will perform differently even when they execute the same algorithm, thus a there is a need to create a scaling factor between the languages to offset any language related performance problems.

The execution time is measured using the .NET Stopwatch class \cite{Stopwatch} which starts from the point of executing the thread which invokes the submissions entry method and ends when the thread ends, either by returning an answer or timing out.


\subsection{Memory Consumption}
The secondary objective is memory consumption. Every problem has the option to restrict memory consumption. This will allow administrators of the system to create problems in which the goal could be to force users to make a trade-off between speed and memory, illustrating real life situations such as optimizing code in order to avoid the unnecessary cost of purchasing more memory.
The tool used to measure the memory consumption (and more) is the Windows Performance Monitor. This tool allows for tracking of Windows processes, and more specifically, tracking of individual resources contained in the .NET CLR.
