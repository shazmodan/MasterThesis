\section{Difficulties and Limitations}

\subsection{Security} \label{subsec:security}
Executing unknown code can be dangerous for several reasons. The code might have unforeseen side effects or even attempt to gain root access (compromise of the entire system). Thus, it was necessary to avert this danger by running the code in a safe and secure sandboxed environment.

In .NET this can be handled through the use of AppDomains which act as a layer of isolation between applications \cite{ApplicationDomains}. AppDomains make it possible to protect the core application from malicious code and exceptions that might occur during the execution of unknown code. This protection comes at the cost of performance since AppDomains are able to load assemblies but cannot unload them. To get rid of a loaded assembly, the entire AppDomain must be unloaded, which is the case with each new submission. CELINE creates a new AppDomain for each submission and then uses proxy objects (remotable objects) called ``Marshals'' \cite{Marshals} to enable communication between the core-AppDomain and the untrusted-AppDomain, in order to get the output that the unknown code produced.

When a new submission enters the AGS, the code is copied to a temporary location on disk and then gets executed with minimal privileges using this AppDomain. An alternate way to handle the security is mentioned in \cite{Suleman}, where user impersonation is used by the core application, essentially letting the operating system handle the security through the use of a minimal privilege user.

There is an exception to the rule about AppDomains being safe from each other. That is when the code that is executing in the untrusted-AppDomain generates a stack overflow exception. This exception will travel back into its parents AppDomain (in this case the core-AppDomain) since it is no longer possible to catch this exception as of .NET version 2.0 and later \cite{StackOverflowException}. In order to avoid crashes caused by stack overflow exceptions, the AGS is instantiated with each new submission by the WCF Service. Thus, the website never experiences any problems, only a timeout of the WCF Service.


\subsection{IronPython Limitations} \label{subsec:ironpython_limitations}
IronPython does not have the same performance as the other languages for a number of reasons. The first reason is that it is 61.6\% slower on average than CPython (version 2.7) \cite{IronPythonPerformance}. The startup time is also significantly slower than the other languages since this language is made compatible using a ScriptEngine. This adds another layer of interpretation (see Chapter \ref{chap:results} for performance results). However it is necessary to mention that IronPython does have the capability to compile the code into an assembly, but with a serious drawback, the assembly's methods and main entry point cannot be invoked like other assemblies since the MSIL is not CLS-compliant \cite{CLSCompliant} and, therefore, cannot be directly accessed from other .NET languages \cite{AccessingPythonCode}.


\subsection{White- and Black-box Testing} \label{subsec:whitebox_blackbox}
White-box testing is a method of testing a software's internal structures as opposed to black-box testing, which tests application functionality/behavior on a higher level (usually done with users in mind). These concepts are usually discussed in relation to Test-Driven Development (TDD) but are also relevant in this context since the code is being tested and there are different ways of doing it.

As mentioned in section \ref{subsec:status_codes}, CELINE is a mixture of both, a grey-box system since it requires the user to define a method that has a pre-defined signature (used as the invocation point), but it does not care about any other implementation details. The choice for this implementation was made because it saved time compared with the more standard approach of using system in and system out (black-box testing) to provide input and output channels.

