\section{Conclusion}
The goal of this thesis was to evaluate the performance of different languages run in the .NET environment. The results indicate that Python has some significant performance problems. One could argue that the problem isn't Python but rather the tests. Sorting for instance is best handled using the built-in functions. However if one was to apply for a job and the task was to implement simple sorting algorithms in order to demonstrate their programming proficiency, I doubt the recruiter would have them use the built-in functions as this would not reveal whether the applicant understands how to implement the algorithm or not.

This thesis concludes that while implementing multiple language compatability in .NET is possible it's not feasible for all languages. The compensation times of the Python code varies greatly depending on how many built-in functions one can use or is allowed to use for solving a problem. One could instead have .NET be the back-end for handling input/output, security and instead of executing the code inside .NET use seperate processes for each language where this process simply executes the Java code using \textit{java.exe} or Python code using \textit{python.exe}. 