\chapter*{Glossary}

\begin{center}
	\begin{tabular} { m{3cm} | m{11cm} }
		\hline
		\textbf{Term}	& \textbf{Description} \\ \hline
		AGS				& Automatic Grading System, a system for evaluating or grading code. \\ \hline
		CIL				& Common Intermediate Language, the lowest level human-readable programming language in .NET Framework. \\ \hline
		CLI				& Common Language Infrastructure, a specification that describes the runtime environment of Microsoft .NET Framework. \\ \hline
		CLR				& Common Language Runtime, is the virtual machine / execution engine of Microsoft .NET Framework. \\ \hline
		CLS				& Common Language Specification, a set of rules and features that a .NET language must implement and understand. \\ \hline
		DLL				& Dynamic-link library, Microsofts implementation of shared libraries. Provides a mechanism to share code and data. \\ \hline
		DLR				& Dynamic Language Runtime, a set of services to enable dynamic language support for the CLR. \\ \hline
		GC 				& Garbage Collection, reclaim memory from objects that are not in use. \\ \hline
		GUI				& Graphical User Interface, an interface that is image oriented rather than text oriented. \\ \hline
		IKVM				& An implementation of the Java for Mono and Microsoft .NET Framework. \\ \hline
		IL				& Intermediate Language, the language of an abstract machine. \\ \hline
		IronPython			& An implementation of the Python programming language targeting the Microsoft .NET Framework and Mono. \\ \hline
		JVM				& Java Virtual Machine, a program that executes byte-code. \\ \hline
		Mono				& An open source project created to enable .NET Framework cross platform compatibility. \\ \hline
		MSIL				& Microsoft Intermediate Language, a synonym for CIL. \\ \hline
	\end{tabular}
\end{center}

