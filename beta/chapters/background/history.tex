\section{History}
The concept of automatic grading is not new. The earliest known system was built in 1960 by Hollingsworth \cite{Hollingsworth}. This system used punch cards to write programs. Using this student-system approach rather than the traditional student-teacher resulted in several benefits. It cut costs considerably for the staff since the time they needed to grade the students work was severely reduced. The students themselves also spent less time on each task, since they were able to have their work graded immediately instead of waiting for a teacher to do it. This system also made it possible to increase the number of students taking the course. It did, however, have some shortcomings. For instance, a student’s program could modify the grader program, making cheating possible.

An article from 2005 \cite{GenerationReview} describes three generations of automatic graders. The first generation systems were those regarded as being built and/or used in the 1960’s and 1970’s. Unsurprisingly, they used code that were close to pure machine code. In order to make them work, it was sometimes necessary to modify both the compiler and the operating system.

The second generation systems (1980-2000) introduced script-based tools. These involved various verification schemes and also asserted that the code was written in a certain way/style (decided by the teacher). Typically these graders involved command-line GUIs. Languages like C and Java were used extensively.

The third generation (2000-) differs from the second generation systems primarily in two ways. They mostly use web based GUI:s and they often include a plagiarism detection system since students sometimes shared code amongst each other. There were some minor issues among these detection systems \cite{Gradebot} \cite{GenerationReview}. If the programming tasks were too easy or if a lecturer had been excessively thorough when describing the homework, the submissions tended to be similar and thus picked up by the plagiarism detection system. Sometimes this made it difficult to distinguish between real plagiarism and the false positives.



