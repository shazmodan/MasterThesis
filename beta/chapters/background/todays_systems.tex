\section{Today's Systems} \label{sec:todays_systems}
Today’s modern systems, such as those in \cite{Gradebot} \cite{Suleman} \cite{GenerationReview}  \cite{Kattis} \cite{Amelung} (considered to be third generation systems) commonly contain a web based front-end together with a general back-end. The back-end’s main purpose is to save submissions to the database and to send the submitted code to the other back-ends (one for each supported language) while preserving system security through sandboxing. The database can be used to inspect submissions. The modern systems mainly differ in their support of different programming languages, and not every one of them contain a plagiarism detection system.

The modern systems also differ from the first generation in that most have adopted a test-driven education paradigm. Students are no longer penalized for attempting to submit more than one solution, in fact, it is encouraged. An article from 1969 \cite{GradingScheme} describes how students were discouraged from ever making a mistake by giving them a lower score for each subsequent submission on the same problem. Students also received an even lower score if the program had not run to completion or if some results were incorrect. The system described (and built) in this thesis adapts the modern test-driven paradigm in the sense that it does not penalize users for multiple submissions or any submission errors.






