\section{Todays Systems} \label{sec:todays_systems}
Todays modern systems, such as those in \cite{Gradebot} \cite{Suleman} \cite{GenerationReview}  \cite{Kattis} \cite{Amelung} (considered to be third generation systems), commonly contain a web based front-end together with a general back-end whos main purpose is to save submissions to the database and to send the submitted code (pipeline) to the other language specific back-ends (one for each language) while preserving system security through sandboxing. The database can be used to inspect specific submissions. The modern systems mainly differ in their support of different programming languages.

The modern systems also differ from the first generation in that most have adopted a test-driven education paradigm. Students are no longer penalized for attempting to submit more than one solution, in fact it's encouraged. An article from 1969 \cite{GradingScheme} describes how student were discouraged from ever making a mistake by giving them a lower score for each subsequent submission on the same problem. Not only that, but students recieved an even lower score if the program didn't run to completion or if some results were incorrect. The system described (and built) in this thesis adapts the modern test-driven paradigm in the sense that it doesn't penalize users for multiple submissions or any submission errors.