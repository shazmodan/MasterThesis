\section{What is automatic code evaluation?}
Automatic code evaluation, or automatic grading, is a computer system that has the ability to judge code. This is usually done in the following steps:
\begin{enumerate}
  \item A user is given a programming task or problem.
  \item The user attempts to solve this problem by writing code.
  \item The user sends this code to the automatic code evaluation system.
  \item The system compiles the code (if needed).
  \item The resulting program is then run by the system with some test data as input.
  \item The system verifies that the program output is correct (or incorrect). 
  \item An answer is then returned to the user indicating the status of the code which he or she submitted (i.e. ``Accepted'' or ``Wrong answer'').
\end{enumerate}
There are some variation to the process above, sometimes more verbose feedback than ``Wrong answer'' is used, often indicating exactly which test went wrong and why (referens ISECON.2006). This is usually done in order for university students to learn more about debugging their code.