\section{What is automatic grading?}
Automatic grading, is a computer system that has the ability to judge code. The process starts with the user being given a programming task or problem to solve. The user then attemps to solve this problem by writing some code which he or she then sends to the automatic grading system. The system compiles the code (if needed) and then runs it. The output generated is then compared with the correct output for that particular problem. If the output matches, the system returns a status message indicating a successful submission (e.g. ``Accepted'') or if the output doesn't match a different message is returned, indicating that something went wrong (e.g. ``Wrong answer''). 

There are some variations to the process above, sometimes more verbose feedback than ``Wrong answer'' is used, often indicating exactly which test went wrong and why \cite{Gradebot}. This is usually done in order for university students to learn more about how to code by debugging/fixing their own code.

Having an automatic grading system results in several benefits \cite{Suleman}. There is no longer a need for humans correcting code in detail (a very time consuming process) since the system acts as a ``judge''. Using this judge also gives greater consistency while evaluating code since all submissions are judged equally. The system can provide instantaneous feedback, making the waiting time considerably shorter, this is particularly true for universities where students no longer have to wait for a teacher or an assistant to grade their homework. Submissions to this system are saved using a database, providing whoever administrates the system a way to trace all interactions with it.
