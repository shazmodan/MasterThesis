\section{What is automatic code evaluation?}
Automatic code evaluation, or automatic grading, is a computer system that has the ability to judge code. The process starts with the user being given a programming task or problem to solve. The user then attemps to solve this problem by writing some code which he or she then sends to the automatic code evaluation system. The system compiles the code (if needed) and then runs it. The output generated by the users code is compared with the correct output for this particular problem. If the output matches, the system returns a status message indicating a successful submission (e.g. ``Accepted'') or if the output doesn't match a different message is returned, indicating that something went wrong (e.g. ``Wrong answer''). 

There are some variations to the process above, sometimes more verbose feedback than ``Wrong answer'' is used, often indicating exactly which test went wrong and why \cite{Gradebot}. This is usually done in order for university students to learn more about how to code by debugging/fixing their own code.