\section{Testing Systems Specifications}
The tests were conducted on two computers. The system specifications can be seen in Table \ref{table:system_specs}. The Macbook Pro runs Windows 7 with the help of Parallels Desktop 8. 

\begin{table}[h]
	\begin{center}
		\begin{tabular} { m{4cm} | m{4cm}  | m{5cm} }
			\hline
			\textbf{} & \textbf{Macbook Pro} & 
			\textbf{Windows 8 Desktop}  \\ \hline

			\textbf{Operating System}		& OSX 10.8.2 
											& Windows 8 Pro 64-bit v6.2 \\ \hline

			\textbf{Processor}				& 2.6 GHz Intel Core i7 
											& 2.4 GHz Intel Core2Quad  \\ \hline

			\textbf{Memory}					& 16 GB 1600 MHz DDR3
											& 4GB 400 MHz DDR3 \\ \hline

			\textbf{Parallels Memory}		& 8 GB dedicated memory 
											& --- \\ \hline

			\textbf{Parallels \# CPU:s} 		& 4 dedicated cores
											& --- \\ \hline
		\end{tabular}
	\end{center}
	\caption{System specifications for the computers used.}
	\label{table:system_specs}
\end{table}
It may appear strange that Parallels have four dedicated cores when the Intel Core i7 processor only contains 4 cores, but this is made possible due to the use of virtual cores (8 in total).

All tests were run on both computers. Test results show that the machine running Windows 8 natively is 2-3 times slower on every test, the reason seem to be hardware related. Due to this fact the results from the native Windows computer will not be presented in this thesis. The results presented in this chapter are from the computer running OSX and should not be viewed in absolute terms but rather in relative terms which is why some tables contain a ratio column (the ratios on the native Windows computer are very similar to the Parallels Windows).



