\subsection{Common Operators}

\begin{figure}[h]
	\centering
	\mbox{
		\subfigure[Test results from addition operator.]{
			\includegraphics[width=0.48\textwidth]{chapters/media/addition.png}
			\label{fig:addition}
		}
		\subfigure[Test results from subtraction operator.]{
			\includegraphics[width=0.48\textwidth]{chapters/media/subtraction.png}
			\label{fig:subtraction}
		}
	}
	\caption{Test results for the addition operator to the left and subtraction to the right.}
	\label{fig:addition_subtraction}
\end{figure}

\begin{figure}[h]
	\centering
	\mbox{
		\subfigure[Test results from multiplication operator.]{
			\includegraphics[width=0.48\textwidth]{chapters/media/multiplication.png}
			\label{fig:multiplication}
		}
		\subfigure[Test results from division operator.]{
			\includegraphics[width=0.48\textwidth]{chapters/media/division.png}
			\label{fig:division}
		}
	}
	\caption{Test results for the multiplication operator on the left and division to the right.}
	\label{fig:multiplication_division}
\end{figure}

\begin{figure}[h]
	\centering
	\includegraphics[width=0.48\linewidth]{chapters/media/modulo.png}
	\caption{Test results from modulo operator.}
	\label{fig:modulo}
\end{figure}

As can bee seen in Figure \ref{fig:addition_subtraction} - \ref{fig:modulo} the time is proportional (linearly) to the number of elements operated on, it takes about 10 times longer to operate on 10 million numbers compared to 1 million.  
