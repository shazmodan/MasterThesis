\begin{foreignabstract}{swedish}

En gemensam utmaning för konsultföretag inom mjukvaruutveckling är att rekrytera rätt personer. Rekryteringsprocessen inom mjukvaruindustrin innefattar vanligtvis flera steg innan ett anställningsavtal undertecknas. En enda anställningsintervju är sällan tillräckligt. Således tenderar intervjun att involvera någon form av test för att se till att den person som söker positionen är kvalificerad för den. Testproceduren är oftast en del av intervjun eller genomförs vid samma tillfälle. Att intervjua sökande som inte är kvalificerade för positionen i fråga kan vara ett slöseri med tid.

Detta slöseri kan minimeras genom att endast intervjua kvalificerade sökande. Inom mjukvaruindustrin säkerställs kvalifikationer vanligtvis genom att låta en sökande lösa programmeringsproblem. Denna process kan automatiseras med hjälp av ett Automaträttningssystem. Sådana system finns redan på flera universitet i dag och används i stor utsträckning i flera programmeringskurser och förekommer även i samband med programmeringstävlingar.

Denna uppsats utvärderar ett sådant system med hänsyn till exekveringshastighet och minneskonsumtion. Den förklarar också hur ett sådant system kan byggas och hur de olika språken presterade. Systemet är unikt i och med att det är byggt med enbart Microsoft .NET Framework och samtidigt klarar av att stödja flera olika programmeringsspråk. De språk som stöds är C\#, Java och Python. Detta stöd möjliggjordes genom användning av en Java byte-kod till CIL kompilator som kallas IKVM och Python stöds genom användning av IronPython.

Resultaten visar att C\# och Java presterar nästan lika med hänsyn till exekveringshastighet, med Java på andraplats. Python verkade ha större prestandaproblem än de andra två. Java konsumerade mest minne av de tre.

Framtida arbete omfattar stöd för fler språk och förbättra systemet med användbarhet i åtanke.

\end{foreignabstract}