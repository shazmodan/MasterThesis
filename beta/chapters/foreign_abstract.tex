\begin{foreignabstract}{swedish}

En gemensam utmaning för konsultföretag inom mjukvaruutveckling är att rekrytera rätt personer. Rekryteringsprocessen inom mjukvaruindustrin innefattar vanligtvis flera steg innan ett anställningsavtal undertecknas. Enbart en anställningsintervju räcker sällan för att avgöra en sökandes lämplighet. Således tenderar intervjun att involvera någon form av test för att se till att sökanden är kvalificerad. Testproceduren är oftast en del av intervjun eller genomförs vid samma tillfälle. Att intervjua sökande som inte är kvalificerade för positionen i fråga kan vara ett slöseri med tid.

Detta slöseri kan minimeras genom att endast intervjua kvalificerade sökanden. Inom mjukvaruindustrin säkerställs kvalifikationer vanligtvis genom att låta sökanden lösa programmeringsproblem. Denna process kan automatiseras med hjälp av ett s.k.Automaträttningssystem. Sådana system finns redan på flera universitet i dag och används i stor utsträckning i flera programmeringskurser och förekommer även i samband med programmeringstävlingar.

Denna uppsats förklarar hur ett sådant system kan byggas med enbart Microsoft .NET Framework och samtidigt stödja ett flertal programmeringsspråk. Uppsatsen utvärderar detta system med hänsyn till exekveringshastighet och minneskonsumtion för att kunna hitta en skalningsfaktor mellan de olika språken som stöds. Skalningsfaktorn är nödvändig för att kunna jämka exekveringstiderna och minnesåtgången mellan språken så att samma implemenation av specifika algoritmer graderas lika oavsett språk. De språk som stöds är C\#, Java och Python. Detta stöd möjliggjordes genom användning av en Java byte-kod till CIL kompilator som kallas IKVM och Python stöds genom användning av IronPython.

Resultaten visar att C\# och Java presterar nästan lika med hänsyn till exekveringshastighet, med Java på andraplats. En skalningsfaktor togs fram som en kompensation för Javas långsammare exekveringstider. Medelvärdet på denna skalningsfaktorn var 1.29. Python verkade ha större prestanda- och minnesproblem än de andra två och ingen skalningsfaktor för detta språk kunde tas fram.

Framtida arbete omfattar stöd för fler programmeringsspråk och förbättra systemet med användbarhet i åtanke.

\end{foreignabstract}
